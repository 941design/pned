% Created 2015-01-11 Sun 22:05
\documentclass[11pt]{article}
\usepackage[utf8]{inputenc}
\usepackage[T1]{fontenc}
\usepackage{fixltx2e}
\usepackage{graphicx}
\usepackage{longtable}
\usepackage{float}
\usepackage{wrapfig}
\usepackage{soul}
\usepackage{textcomp}
\usepackage{marvosym}
\usepackage{wasysym}
\usepackage{latexsym}
\usepackage{amssymb}
\usepackage{hyperref}
\tolerance=1000
\providecommand{\alert}[1]{\textbf{#1}}

\title{Bedienhandbuch PnEd}
\author{Markus Rother (8544832)}
\date{\today}
\hypersetup{
  pdfkeywords={},
  pdfsubject={},
  pdfcreator={Emacs Org-mode version 7.9.3f}}

\begin{document}

\maketitle


In diesem Bedienhandbuch wird der \emph{PnEd} - ein einfacher Petri-Netz
Editor vorgestellt.  Dieser, sowie die hiermit vorliegende
Dokumentation sind als Aufgabenbearbeitung des Grundpraktikum
Programmierung (Kurs 01584) im Wintersemester 2014/15 an der
Fernuniversität in Hagen entstanden.  Der Editor wurde von mir (Markus
Rother) in Java implementiert, und mit dem Namen \emph{PnEd} versehen.

Diese Dokumentation skizziert meine wichtigsten Designentscheidungen
und wie sie entstanden sind.  Anschließend wird die Grundlegende
Architektur der Software vorgestellt, also die Strukturierung der
\verb|packages| und der wichtigsten Klassen. Die Bedienanleitung
beschreibt detailliert die Verwendung des Editors.  Abschließend wird
das größte bekannte Problem kurz beschrieben.
\pagebreak
\tableofcontents
\pagebreak

\section{Design des PnEd}
\label{sec-1}
\subsection{Loose Kopplung und Single-Responsibility-Prinzip}
\label{sec-1-1}


   Zu Beginn der Bearbeitungsphase stürzte ich mich kopfüber in die
   Implementierung eines Datenmodells. Als ich dann begann, an der
   Benutzeroberfläche zu arbeiten, stellte sich rasch heraus, dass die
   beiden nicht so recht zusammenpassen wollten. Ich musste das
   Datenmodell häufig an die Anforderungen der Benutzeroberfläche
   anpassen.  Außerdem war vieles am Datenmodell unnötig.

   Es bestand ganz offensichtlich eine sehr starke Kopplung zwischen
   meinem Datenmodell und der Benutzeroberfläche, die mir missfiel.
   Ich entschied mich, das Datenmodell vollständig zu ignorieren und
   zunächst die Oberfläche zu schreiben.  Ich wollte bei der Erstellung der
   Oberfläche so wenig Annahmen wie möglich über das zugrunde liegende
   Datenmodell machen.

   Dazu sollte die Verbindung zwischen Datenmodell und der
   Benutzeroberfläche über Ereignisse (events) stattfinden.  Ähnlich
   einer Client-Server Architektur sollte die Benutzeroberfläche mit
   dem Datenmodell mittels Nachrichten sprechen.  Ich bräuchte dann
   lediglich einen Verbindungsstrang, über den diese Nachrichten,
   Ereignisse oder Anfragen kommuniziert würden.

   Die Benutzeroberfläche selbst sollte dabei frei von jeglicher Logik
   bleiben, zum Beispiel nicht für das Berechnen des Netzes bei
   Schaltung von Transitionen verantwortlich sein. Und das Datenmodell
   sollte keinerlei Kenntnis des Zustands der Visualisierung haben,
   nicht einmal, welche Knoten gerade ausgewählt wären.

   Mit diese puristischen herangehensweise gelang es mir zu meiner
   eigenen Überraschung, die Benutzeroberfläche fast vollständig -
   sozusagen als eine Hülle - zu implementieren, ohne mich im
   geringsten um die Logik des Petri-Netzes zu kümmern.  Zu meiner
   noch größeren Überraschung war das Anschließende implementieren des
   Datenmodells eine Arbeit von kaum zwei Stunden.  Noch einfacher war
   die Anbindung des vorgegebenen PNMLParsers, da auch dieser wie die
   Benutzeroberfläche lediglich Nachrichten verschicken musste, also
   etwa für das Erstellen eines Knotens eine Nachricht mit Typ, Namen
   und Koordinaten.  Diese Nachricht würde dann sowohl vom Datenmodell
   als auch von der Benutzeroberfläche entsprechend interpretiert.

   Kritisch resümierend kann ich folgendes festellen: Auch wenn ich
   mir im Verlauf des Projekts durch mein Vorgehen zahlreiche andere
   Probleme einhandelte, halte ich meine Entscheidung für gut. Ich
   habe erfahren, wo die Grenzen einer solchen Trennung liegen, ich
   konnte mich mit Nebenläufigkeit- und musste mich mit einigen
   Details des \verb|swing| frameworks befassen. Es wäre sicherlich
   vieles einfacher gewesen, aber bei weitem nicht so erkenntnisreich.
\subsection{Aufbau der Pakete}
\label{sec-1-2}


   Das Projekt ist in 23 Pakete unterteilt.  Diese können weiter in
   drei größere Bereiche gruppiert werden:

\begin{enumerate}
\item Pakete, deren Klassen die Kernfunktionalität des \emph{PnEd}
      auszeichnen. Ohne diese Klassen funktioniert nichts, sie
      funktionieren aber ohne andere Klassen des Editors.
\item Pakete, deren Klassen die Funktionalität der Benutzeroberfläche
      bereitstellen.  Sie stehen ebenfalls weitestgehend für sich,
      haben also kaum Abhängigkeiten zur ersten Gruppe.
\item Pakete, deren Klassen Hilfsfunktionalität bereitstellen. Auch
      diese sind frei von Abhängigkeiten zu den oberen beiden.
\end{enumerate}

   Im Anschluß werden die wichtigsten Pakete
   aus 1. und 2. vorgestellt.
\subsubsection{.pned.core.model}
\label{sec-1-2-1}

      In diesm Paket befinden sich die Schnittstellenklassen des
      Datenmodells. Es handelt sich hierbei um das von der
      Logik-Schicht verwendete Datenmodell, welches nicht von der
      Benutzeroberfläche benötigt wird. Das Paket beinhaltet keine
      Implementierungen.
\subsubsection{.pned.core}
\label{sec-1-2-2}

     In diesm Paket befinden sich je eine mögliche Implementierungen
     der Schnittstellenklassen aus \verb|.pned.core.model|. Außerdem
     liegen hier die Ausnahmen (\verb|exceptions|), die in Klassen
     dieses Pakets auftreten können.
\subsubsection{.pned.gui.core.model}
\label{sec-1-2-3}

    Analog zu \verb|.pned.core.model| befinden sich in diesem Paket
    diejenigen Schnittstellentypen, die für die Visualisierung
    notwendig sind, also etwa das zustandsabhängige Aussehen der
    Knoten und Kanten.
\subsubsection{.pned.gui.core}
\label{sec-1-2-4}

    Hier befinden sich Implementierungen der Schnittstellen aus
    \verb|.pned.gui.core.model|.
\subsubsection{.pned.gui.components}
\label{sec-1-2-5}

    In diesem Paket und seinen Unterpaketen sind die eigentlichen
    Komponentenklassen abgelegt, also vor allem Erweiterungen von
    \verb|javax.swing| Objekten für Knoten, Kanten und andere
    Elemente, außerdem das Hauptfenster, Dialogfenster, Menus, sowie
    Listener-Klassen welche den Arbeitsfluß der Benutzeroberfläche
    steuern.  So gibt es zum Beispiel unter
    \verb|.pned.gui.components.listeners| ein Listener-Objekt, welches
    für die Auswahl von Knoten zuständig ist.
\subsubsection{.pned.control und .pned.gui.control}
\label{sec-1-2-6}

    In diesen beiden Paketen und ihren Unterpaketen befinden sich die
    Klassen die für die eingangs geschilderte Art der Kommunikation
    durch Nachrichten zuständig sind.  Diese Pakete bilden sozusagen
    das Herz des \emph{PnEd}.  Hier liegen jeweils drei Unterpakete für
    Ereignis- oder Nachrichtenarten:
    
\begin{description}
\item[.commands] Hier liegen Nachrichten-Klassen, welche
                   Veränderungen anstoßen, sozusagen \emph{befehlen}
                   sollen. In der Regel werden diese durch
                   Nutzeraktionen angestoßen.
\item[.events] Hier liegen Nachrichten-Klassen, welche Ereignisse
                 ausdrücken sollen, die nicht \emph{befohlen} wurden,
                 sondern vielmehr aufgrund vorangegangener \emph{Befehle}
                 enstanden sind, also aufgrund der Funktionsweise von
                 Petri-Netz und Oberfläche ausgelöst wurden.
\item[.requests] Hier liegen Nachrichten-Klassen, welche Antworten
                   erwarten. Derzeit gibt es nur zwei Nachrichten für
                   die das zutrifft: 1. Die Anfrage nach einem Namen
                   für ein neues Element, und 2. Die Anfrage nach
                   einem Element für einen Namen. Letztere wird
                   lediglich von der Oberflächenschicht benötigt.
\end{description}

    In diesen Paketen liegen außerdem jeweils Listener-Klassen für die
    beschriebenen Ereignisse. Und schließlich liegen in den Paketen
    \verb|.pned.control| und \verb|.pned.gui.control| die Klassen
    \verb|EventBus| und \verb|PnEventBus| - dem eigentlichen
    Nachrichtenkanal.

    Wie eingangs beschrieben, findet der Nachrichtenaustausch über den
    \verb|EventBus| statt.  Gültige Eingaben und Ereignisse die eine
    Veränderung beschreiben werden als ein Ereignis
    instanziert. dieses wird dann vom Aufrufer, z.B. einem Listener,
    an den \verb|EventBus| weitergereicht. Dieser implementiert selbst
    die entsprechende Schnittstelle. Seine Aufgabe besteht nun
    lediglich darin, das Ereignis an die bei ihm registrierten
    Listener zu verteilen. Man könnte auch von einem Multiplexer
    sprechen.
    
    Eine Konsistenzprüfung erfolgt in der Regel nicht. Es wird also
    nicht festgestellt, ob Ereignisse in einer bestimmten Reihenfolge
    bedient werden können.  Auch eine Fehlerbehandlung wird nicht
    zurückgereicht.  Das verwendete Protokoll ist sozusagen
    \emph{verbindungslos}. Das birgt natürlich potentielle Probleme, war
    aber eine bewußte Entscheidung zugunsten looser Kopplung.  Die
    Darstellung ist letztlich nicht für die Konsistenz des
    Datenmodells verantwortlich und umgekehrt!
\subsubsection{Weiter Pakete}
\label{sec-1-2-7}

    
    Es folgt lediglich eine Kurzbeschreibung der restlichen Pakete.

\begin{description}
\item[.concurrent] Hilfsklassen für die Behandlung von Nebenläufigkeit.
\item[.pned.gui.actions] Unterklassen von
         \verb|javax.swing.AbstractAction|, welche Nutzereingaben
         abstrahieren.
\item[.pned.io] Klassen, die für die Ein-, Ausgabe von Petri-Netzen
                  zuständig sind. Hierzu gehört der vorgegebene
                  \verb|PNMLParser| sowie Klassen zur Serialisierung
                  des Petri-Netz Modells.
\item[.pned.util] Hier liegen Hilfsklassen, die für den Betrieb des
                    \emph{PnEd} nicht nötig sind, aber während der
                    Entwicklung hilfreich waren, z.B. ein Logger.
\item[.swing] Eine Ansammlung von eigenständigen Erweiterungen des
                \verb|swing| frameworks. Hier liegt z.B. ein
                selbst-validierendes Eingabefeld, wie es im Editor
                verwendet wird.
\item[.util] Weitere, allgemeine Hilfsklassen.
\end{description}

\pagebreak
\section{Bedienung des \emph{PnEd}}
\label{sec-2}
\subsection{Verwaltung der Dateien}
\label{sec-2-1}
\subsubsection{Erstellen eines neuen Petri-Netzes}
\label{sec-2-1-1}

    Um ein neues Petri-Netz zu erstellen, wählen Sie in der Menüleiste
    $\Rightarrow$ \textbf{File} $\Rightarrow$ \textbf{New}.  Diese Aktion verwirft das
    derzeit editierte Netz unwiderruflich und ohne Nachfrage.
\subsubsection{Import/Öffnen}
\label{sec-2-1-2}

    Zum Öffnen einer bereits vorhandenen Datei wählen Sie in der
    Menüleiste $\Rightarrow$ \textbf{File} $\Rightarrow$ \textbf{Import}. Wählen Sie die
    zu öffnende Datei aus Ihrem Dokumentenordner und bestätigen Sie
    über den \textbf{Open} Druckknopf.  Die voreingestellte Dateiendung ist
    \verb|.pnml|, es lassen sich jedoch beliebig benannte Dateien
    öffnen.  Geladen werden können Dateien eines nicht näher
    spezifizierten PNML-Formats.  Kann eine Datei nicht geladen
    werden, bleibt die Aktion folgenlos.  Auch das Öffnen einer einer
    Datei verwirft das derzeit editierte Netz unwiderruflich und ohne
    Nachfrage.
\subsubsection{Export/Speichern}
\label{sec-2-1-3}

    Erstellte Petri-Netze können Sie exportieren über Menüleiste
    $\Rightarrow$ \textbf{File} $\Rightarrow$ \textbf{Export}, der Anwahl des
    Speicherortes und Bestätigen über den \textbf{Save} Druckknopf. Die
    Dateien werden in einem PNML-Format abgespeichert.  Wird eine
    existierende Datei als Ziel des Speichervorgangs ausgewählt, so
    wird diese ohne weitere Nachfrage überschrieben.  Es besteht keine
    Garantie, dass das gespeicherte Format eines zuvor geladenen
    Netzes binärkompatibel ist, nicht einmal wenn dieses Netz mit dem
    \emph{PnEd} erstellt wurde.
\subsection{Erstellen/Löschen von Knoten und Kanten}
\label{sec-2-2}


   Innerhalb des Rasterfelds können Sie beliebig erweiterbar Knoten
   und Kanten setzen, das editierbare Feld vergrößert sich automatisch
   mit dem Setzen von Knoten bei Bedarf. Die Vergrößerung des
   Rasterfeldes geschieht ausschließlich auf diese Weise, also wenn
   neue Knoten in der Nähe des rechten oder unteren Randes gesetzt
   werden.  Über die Bildlaufleisten kann dann der Sichtbare Bereich
   verändert werden.
\subsubsection{Erstellen und Platzieren von Stellen}
\label{sec-2-2-1}

    Zum Erstellen einer Stelle wählen Sie in der Menüleiste
    $\Rightarrow$ \textbf{Edit} und aktivieren das Optionsfeld vor \textbf{Create     place}, so dass das Suffix \textbf{(default)} erscheint. Alternativ
    können Sie ein Kontextmenü über einen Rechtsklick im Rasterfeld
    aufrufen und diese Einstellung dort vornehmen. Wenn Sie
    anschließend mit der linken Maustaste in das Rasterfeld klicken
    erhalten Sie automatisch eine Stelle. 

    Solange das Optionsfeld bei \textbf{Create place} gesetzt bleibt,
    generiert jeder Linksklick im Rasterfeld eine neue Stelle. Dies
    ist auch der voreingestellte Wert, so dass bei neu geöffnetem
    \emph{PnEd} zunächst Stellen erzeugt werden.
\subsubsection{Erstellen und Platzieren von Transitionen}
\label{sec-2-2-2}

    Zum Erstellen einer Transition wählen Sie in der Menüleiste
    $\Rightarrow$ \textbf{Edit} und aktivieren das Optionsfeld vor \textbf{Create     transition}, so dass das Suffix \textbf{(default)} erscheint. Alternativ
    können Sie ein Kontextmenü über einen Rechtsklick im Rasterfeld
    aufrufen und diese Einstellung dort vornehmen. Wenn Sie
    anschließend mit der linken Maustaste in das Rasterfeld klicken
    erhalten Sie automatisch eine Transition.

    Solange das Optionsfeld bei \textbf{Create transition} gesetzt bleibt,
    generiert jeder Linksklick im Rasterfeld eine neue Transition.
    Neu erzeugte Transitionen werden grün dargestellt, wodurch ihre
    Aktivierung kenntlich ist.
\subsubsection{Verschieben von Stellen und Transitionen}
\label{sec-2-2-3}

    Einzelne Stellen oder Transitionen können jederzeit mit dem
    Mauszeiger bei gehaltener linker Maustaste verschoben werden.

    Zum Verschieben mehrerer Knoten ziehen Sie bei gehaltener linker
    Maustaste ein Auswahlfeld über die zu verschiebenden Stellen
    bzw. Transitionen. Die Knoten sind dann durch eine rote Umrahmung
    markiert. Ziehen sie nun einen beliebigen Knoten mit dem
    Mauszeiger bei gehaltener linker Maustaste, um alle ausgewählten
    Knoten zu verschieben.
\subsubsection{Löschen von Stellen und Transitionen}
\label{sec-2-2-4}

    Zum Löschen von Knoten ziehen Sie bei gehaltener linker Maustaste
    ein Auswahlfeld über die zu löschenden Stellen
    bzw. Transitionen. Die Knoten sind dann durch eine rote Umrahmung
    markiert. Wählen Sie über die Menüleiste oder das Kontextmenü
    $\Rightarrow$ \textbf{Edit} $\Rightarrow$ \textbf{Remove selected nodes} - die
    markierten Objekte werden ohne Nachfrage und unwiderruflich
    gelöscht.  

    Beim Löschen von Knoten werden automatisch auch die an den Knoten
    verankerten Kanten entfernt.
\subsubsection{Erstellen von Kanten}
\label{sec-2-2-5}

    Um eine Kante zu erstellen klicken Sie mit der linken Maustaste
    zweimal hintereinander auf einen Knoten.  Es erscheint eine Kante,
    die Ihrem Mauszeiger folgt.  Eine Rotfärbung kennzeichnet, dass
    die Kante am Ort Ihres Mauszeigers nicht verankert werden kann,
    eine Grünfärbung, dass eine Verankerung möglich ist.

    Wenn Sie während des Erstellens einer ungültigen Kante mit der
    linken Maustaste klicken, wird die Kante verworfen.  Wenn sie auf
    einem gültigen Zielknoten zweimal hintereinander klicken, wird die
    Kante am Zielknoten verankert.
\subsubsection{Löschen von Kanten}
\label{sec-2-2-6}

    An einem Knoten können alle eingehenden und alle ausgehenden
    Kanten gelöscht werden.  Eine Auswahl und Löschung einzelner
    Kanten ist nicht möglich.
    
    Um eingehende Kanten an einem Knoten zu entfernen, ziehen Sie bei
    gehaltener linker Maustaste ein Auswahlfeld über die Stellen
    bzw. Transitionen deren eingehende Kanten entfernt werden sollen.
    Die Knoten sind dann durch eine rote Umrahmung markiert.  Über die
    Menüleiste oder das Kontextmenü wählen sie $\Rightarrow$ \textbf{Edit}
    $\Rightarrow$ \textbf{Remove incoming edges}. Die eingehenden Kanten werden
    entfernt.

    Um ausgehende Kanten zu entfernen, verfahren sie wie oben und
    wählen stattdessen $\Rightarrow$ \textbf{Edit} $\Rightarrow$ \textbf{Remove outgoing     edges}. Die ausgehenden Kanten werden entfernt.
\subsection{Markieren von Stellen}
\label{sec-2-3}


   Neu erstellte Stellen sind ohne Markierung, bzw. mit einer
   sogenannten Nullmarke versehen. Um einer Stelle eine Markierung
   zuzuweisen, klicken Sie mit der rechten Maustaste in eine Stelle.
   Ein kleines Textfeld erscheint in welches Sie eine Zahl eingeben
   können. Durch Bestätigen mit der \verb|Enter| oder \verb|Return|
   Taste wird der Stelle diese Markierung zugewiesen. Die Eingabe kann
   mit der \verb|ESC| Taste abgebrochen werden.  Eine ungültige
   Eingabe wird durch roten Text gekennzeichnet.

   Die Markierung einer Stelle wird entweder durch einen schwarzen
   Punkt oder eine Zahl \verb|>2| gekennzeichnet. Nullmarken werden
   nicht ausgezeichnet.
\subsection{Aktivierung von Transitionen}
\label{sec-2-4}


   Aktivierte, grün gefärbte Transitionen können geschaltet werden.
   Das Schalten erfolgt durch Klicken der rechten Maustaste auf der
   aktivierten Transition.  Transitionen können nur einzeln geschaltet
   werden.
\subsection{Beschriftung von Knoten}
\label{sec-2-5}


   Um eine Stelle oder Transition zu beschriften, klicken Sie mit der
   rechten Maustaste in das Beschriftungfeld über einem Knoten. Es
   erscheint ein editierbares Textfeld in das Sie einen neuen Namen
   schreiben können. Durch Bestätigen mit der \verb|Enter| oder
   \verb|Return| Taste wird dem Knoten der neue Name zugewiesen. Die
   Eingabe kann mit der \verb|ESC| Taste abgebrochen werden.  Eine
   ungültige Eingabe wird durch roten Text gekennzeichnet.

   Die Beschriftungen können mit gehaltener linker Maustaste relativ
   zu ihren Knoten verschoben werden.
\subsection{Ändern der Größe von Knoten, Marken und Kanten}
\label{sec-2-6}


   Um die Größe der Knoten, Schriftgröße der Markierungen oder die
   Größe der Pfeilspitzen zu verändern, wählen Sie in der Menüleiste
   $\Rightarrow$ \textbf{Preferences} $\Rightarrow$ \textbf{Settings}. Sie können nun
   über die Eingabe im Textfeld oder Verschieben des Reglers die
   jeweilige Größe ändern.
\section{Bekannte Probleme}
\label{sec-3}
\subsection{OutOfMemoryException bei NodeRequests}
\label{sec-3-1}


   Requests - Ereignisse die eine Antwort erwarten und Asynchron
   gestartet werden, Überfluten den Threadpool.  Diese Problem tritt
   bereits bei relativ kleinen Netzen auf, da die voreingestellte
   Lebensdauer abgehandelter Threads recht hoch ist.  Pro Listener
   Aufruf wird ein neuer Thread gestartet, der weiterlebt nachdem der
   Aufruf (erfolglos) endet. Das Problem ist lösbar, indem ein eigener
   Threadpool implementiert wird, welcher Threads wiederverwendet, und
   beendete Threads aufräumt. Die Stelle im Code:
   \verb|de.markusrother.pned.gui.control.PnEventBus.requestNode(NodeRequest)|
   

    

\end{document}
